\documentclass{conf/FusinaClass}



\title{Fusina - Fun, Simple, Narratif}
\author{Lionel "Armaklan" Zuber et Sébastien "Segle" Gelé}
\begin{document}
\pagestyle{couv}
    ~~\\
    \clearpage
    \pagestyle{plain}
\maketitle
\clearpage
\tableofcontents
\clearpage

\chapter*{Crédits}

\vskip 140pt

\begin{center}

\textbf{Conception}

Lionel "Armaklan" Zuber

\vskip 80pt

\textbf{Avec la participation de}

Lionel "Armaklan" Zuber

Simon "Angeldust" Li

Sébastien "Segle" Gélé

Léandre "Salanael" Bernier

\textbf{Et remerciement à \ldots}

\end{center}

\clearpage

\chapter*{Licence}

Fusina est un \textbf{système générique de jeu de rôle}, sous licence \textbf{Creative Commons BY-SA} dont le détail est à cette adresse : \url{http://creativecommons.org/licenses/by-sa/3.0/fr/deed.en}.

Cette licence vous autorise à diffuser, modifier ce document, du moment que vous citez son auteur et que vos modifications soient sous cette même licence. C'est une licence virale.

Les icônes des encadrés font partie du pack d'icônes "ecqlipse 2" de chrfb, dont la galerie DeviantArt se trouve ici : \url{http://chrfb.deviantart.com/gallery/}. Elles sont en Creative Commons By : \url{http://creativecommons.org/licenses/by/3.0/deed.fr}.

Les bordures et images entourant les titres des chapitres sont des réalisations d'Ivy-Poison (Random Borders et Dirty Lines) sur Deviantart, dont la galerie se trouve là : \url{http://ivy-poison.deviantart.com/gallery/}.

Toutes ces images sont la propriété de leurs auteurs respectifs et sont utilisés ici dans le cadre des autorisations d'utilisation indiquées par les auteurs. Avant toute réutilisation, assurez-vous de bien respecter les conditions d'utilisation de leur travail.

\bigskip

La dernière version de Fusina est disponible à l'adresse suivante :

%\medskip

\begin{center}
  \Large\url{http://fusina-jdr.org}
\end{center}

%\bigskip

\vfill

\begin{center}
  \Large Bonne lecture !
\end{center}


\vskip 20pt

\clearpage

\chapter*{Guide de Lecture}

Pour faciliter la lecture, certaines zones ont été ressorties sur fond coloré. Ces zones illustrent ce qui est écrit dans le reste du document. Un petit logo en haut à droite de ces zones, ainsi que leur couleur de fond, indiquent leur type.

\exemple{Exemple}{
    Ce type de zone indique une ou plusieurs mises en situation illustrant un point (de règle ou non) défini au dessus.
}

\note{Note de conception}{
    Ce type de zone indique une note de conception. Les notes de conceptions servent à expliquer pourquoi nous avons fait tel ou tel choix, tant au niveau des règles que de l'univers. Elles permettent au lecteur de mieux sentir l'objectif présent derrière un élement.
}

\option{Règle optionnelle}{
    Ce type de zone indique des règles facultatives, pouvant mieux convenir que les règles "de base" dans certains types de parties par rapport à l'univers choisi par le MJ pour ces parties.
}

\remarque{Conseil}{
    Ce type de zone contient des remarques, des astuces ou des avertissements pour que le futur MJ puisse éviter certains problèmes que nous avons nous mêmes déjà rencontrés. Il s'agit principalement de Conseil donné au MJ de manière à l'orienté et à l'aider à utiliser Fusina dans l'optique que nous lui avons choisit.
}

\clearpage

\part{Fusina !}

Fusina est un système de jeu de rôle dont les objectifs sont d'être Simple, Fun, et Narratif.

Simple car nous pensons que le système ne doit pas parasiter l'ambiance de la table. L'histoire et les scènes doivent rester fluides. A aucun moment le maître du jeu ne doit avoir besoin de plonger dans son livre de règles. Le système doit donc être facilement mémorisable pour tout le monde, maître comme joueur.

Fun car nous voulons que les joueurs prennent plaisir à utiliser Fusina. La composante ludique est essentielle dans le jeu de rôle, il est donc important que les joueurs aient des mécaniques sur lesquelles influer, avec lesquelles s'amuser.

Et Narratif car nous pensons que, malgré tout, le plus important reste les personnages et l'histoire. Fusina est conçu pour aider les joueurs à vivre des aventures excitantes. Le système met en avant les personnages et leurs spécificités, ce qui les rend uniques et intéressants.

Contrairement aux apparences, Fusina n'est pas un système générique. Si vous désirez exploiter ce système pour un nouvel univers, il vous faudra l'adapter largement, faire des choix, modifier des élements existants du système. Tel quel, Fusina vous conviendra pour un One-Shot, mais pas pour une longue campagne. Le système de jeu mérite d'être transformé pour correspondre d'avantage à votre univers, c'est d'ailleurs ce que nous faisons pour les différents univers proposés en téléchargement sur le site Fusina.

Fusina est un système en licence Creative Commons. Vous avez le droit de l'utiliser, de le reproduire, de le modifier et même de commercialiser un univers l'utilisant. La seule contrainte : indiquer où trouver le système d'origine.

Je vous souhaite maintenant une bonne lecture et un bon jeu !

\chapter{Les principes du Système}
Ce chapitre présente les principaux élements du système de jeu, vous donnant ainsi un aperçu de ce qui vous attend. Leurs utilisations seront détaillées par la suite.

\section{Les dés}
Fusina utilise différents types de dés très connus des rolistes : à 4 faces, à 6 faces, à 8 faces, à 10 faces et à 12 faces. Dans le livre nous utiliserons la notion suivante "Dx" ou x représente le nombre de face du dé à utiliser. 

Nous utiliserons également une autre notation faisant intervenir plusieurs dés, comme par exemple : D4 + D8. Ce type de notation signifie qu'il faut lancer deux dés, ici un dé à 4 faces et un dé à 8 faces. Vous devrez ensuite conserver le meilleur des résultats (et non la somme).

Nous appellerons "échelon" le passage d'un dé vers le dé supérieur. Par exemple, la différence entre un D4 et un D6 est d'un échelon. La différence entre un D8 et un D12 est de deux échelons.

\section{Les caractéristiques}
Les caractéristiques représentent les capacités naturelles du personnage. Est-il fort, est-il intelligent ? Ce sont les caractéristiques qui répondront à ces questions.

\begin{itemize}
\item Physique : représente les capacités physiques (force, endurance, agilité) du personnage
\item Intellect : représente les capacités de mémoire et de raisonnement du personnage
\item Social : représente les capacités de communication et le charisme du personnage
\item Ame : représente la force d'âme, la volonté et la chance du personnage
\item Influence : représente le niveau social du personnage, ses moyens financiers, ses contacts, ...
\end{itemize}

Les caractéristiques ont une valeur qui va en général de D4 à D12. 

\section{Les compétences}
Les compétences représentent ce que le personnage a appris à faire durant sa vie. Quels enseignements a-t-il suivis ? Quelles professions a-t-il pratiquées ?

Les compétences ont une valeur qui va de D4 à D12.

\section{Les équipements}
L'équipement d'un personnage représente ses possessions, les objets qu'il possède. Dans le système Fusina, la plupart des équipements n'ont pas d'effet réel, ce sont des artifices purement narratifs. 

\section{Les traits}
Les traits sont des adjectifs ou des phrases qui caractérisent le personnage. Ils peuvent définir son apparence, son histoire, son caractère. Tous ce qui rend le personnage particulier, ce qui le fait sortir du commun des mortels, est un trait. Les traits ne sont pas exclusivement les aspects positifs de votre personnage, le négatif fait aussi parti de son charme !

En jeu, les traits peuvent être utilisés comme bonus sur certaines de vos actions. Vous pourrez également choisir de vous pénaliser en mettant en avant l'aspect négatif de l'un de vos traits. Les traits vous seront utiles pour rendre votre personnage unique et mettre en scène autant ses bons que ses sombres cotés.

\section{Résolution d'une action}
Le système de jeu intervient quand un personnage tente une action qu'il n'est pas sûr de réussir ou d'échouer. La résolution d'une action prend en compte les caractéristiques et les compétences. Le joueur lancera les dés correspondant à ces deux élements. Les traits pourront ensuite intervenir comme bonus pour l'aider à accomplir des exploits plus importants.


\part{Création de personnage}
\chapter{Concept}
Premier étape et sûrement la plus importante de la création : définir le concept de votre personnage. Que fait-il ? Qu'est-ce qui le motive ? A quoi il ressemble ? Peut-être avez vous en tête un personnage de film ou de roman qui vous inspire ? 

Si vous n'avez pas d'idée, prenez le temps d'en discuter avec votre maître du jeu. Il vous sera impossible de continuer la création sans avoir au moins une vague idée de ce que vous voulez créer.

\chapter{Les Traits}
Votre personnage est un être unique avec ses forces et ses faiblesses, ses particularités, ses motivations et même son histoire propre. Même si d'autres auront sûrement les mêmes caractéstiques ou compétences, ils seront bel et bien différents. Les traits servent à représenter cette différence, ce qui vous rend hors du commun.

Lors de la création du personnage, vous devez commencer par choisir ses traits. Un personnage standard en possède six. Toutefois, selon l'univers et l'ambiance qu'il vise, le maître du jeu pourra vous en demander un nombre différent.

\exemple{Quelques traits}{

\begin{itemize}
\item Forte tête
\item Montagne de muscles
\item Membre de la haute société
\item Aime tout ce qui explose
\item Ne peut pas tuer d'être humain
\item Hors-la-loi
\item Incapable de garder un secret
\item Ne sait pas dire non à une femme
\end{itemize}

}

Faites attention à ne pas choisir que des traits fondamentalements positifs. Les mauvais cotés de votre personnage auront également leur importance une fois en jeu. N'hésitez donc pas à prendre des traits négatifs, ou des traits ayant à la fois leurs avantages et défauts.

\chapter{Les Caractéristiques}
Les caractéristiques représentent les capacités naturelles du personnage. Est-il fort, est-il intelligent ? Ce sont les caractéristiques qui répondront à ces questions.

Les caractéristiques ont une valeur qui va de d4 à d12. Elle est évaluée sur cette échelle :

\begin{itemize}
\item D4 : Faible
\item D6 : Standard
\item D8 : Supérieur à la moyenne
\item D10 : Excellent
\item D12 : Héroïque
\end{itemize}

Pour déterminer le score d'une caractéristique, nous allons définir, pour chaque trait, une tendance. Elle représente la caractéristique qui semble le plus proche de l'utilisation de ce trait. Le score d'une caractéristique dépend du nombre de traits associés.

\begin{itemize}
\item 0 trait : D4
\item 1 trait : D6
\item 2 traits : D8
\item 3 traits : D10
\item 4 traits : D12
\end{itemize}

Par exemple, "Forte tête" indique une forte volonté du personnage, la tendance de ce trait est donc l'âme. "Montagne de muscles" serait plutôt physique, ...

Il peut arriver que certains traits n'aient pas de tendance : soit ils sont tout simplement négatifs, soit vous ne voyez pas à quoi le rattacher. Dans ce cas, vous pouvez choisir librement la caractéristique qui pourra augmenter d'un échelon. Cette possibilité doit bien sur être exploitée en accord avec le maitre du jeu.

\chapter{Les Compétences}
Les compétences représentent ce que le personnage a appris à faire durant sa vie. Quels enseignements a-t-il suivis ? Quelles professions a-t-il pratiquées ? 

Les compétences ont une valeur qui va de d4 à d12 selon l'échelle suivante : 

\begin{itemize}
\item d4 : Vagues connaissances
\item d6 : Amateur
\item d8 : Professionnel
\item d10 : Expert
\item d12 : Grand Maître
\end{itemize}

Le joueur dispose de 25 points à dépenser pour inscrire des compétences sur sa feuille. Les compétences sont totalement libres. Le niveau de la compétence dépend du nombre de points investis :

\begin{itemize}
\item D4 : 1 points
\item D6 : 3 points
\item D8 : 6 points
\item D10 : 10 points
\item D12 : 15 points
\end{itemize}

\exemple{Exemples de compétences}{

\begin{itemize}
\item Comédien
\item Fin bretteur
\item Astropilote
\item Pratiquant Psy
\end{itemize}

}

\chapter{L'équipement}
Le joueur constitue ensuite une liste d'équipement qu'il désire avoir et qu'il est logique que son personnage ait.

Pour chaque équipement, le MJ a 3 solutions : 

\begin{itemize}
\item Accepter l'équipement : l'équipement est logique pour le personnage et ne requiert pas d'être particulièrement riche. 
\item Refuser l'équipement : au contraire, l'équipement peut être totalement illogique pour le personnage (équipement que le personnage ne sait pas utiliser), ou coûter excessivement cher.
\item Demander un jet d'Influence : l'influence est utilisée pour simuler les contacts du personnages, ainsi que ses ressources. Le jet d'influence permet de savoir si le personnage a pu mettre la main sur l'équipement en question, et s'il a dû s'endetter pour le faire.
\end{itemize}

Le MJ peut utiliser l'une ou l'autre de ces conditions, à sa propre convenance. Nous incitons toutefois à utiliser le jet d'influence pour tous les cas réèllement litigieux.

\exemple{Quelques équipements...}{

\begin{itemize}
\item Dague
\item Matériel d'escalade
\item Tenue de camouflage
\item Canon plasma
\end{itemize}

}


\part{La Résolution d'une action}
La résolution d'une action intervient quand la réussite des actes d'un personnage est incertaine. Si l'échec ou la réussite sont obligatoires, nul besoin de mécanique ! Quand un doute demeure par contre, l'utilisation du système de jeu s'impose alors. Le système décrit ci-dessous vaut pour tous les actes, qu'ils soient physiques, sociaux ou intellectuel.

\chapter{En résumé}
La résolution d'une action se déroule comme suit : 

\begin{itemize}
\item Le joueur indique l'action qu'il veut effectuer.
\item Le MJ indique au joueur la caractéristique et la compétence appropriées.
\item Le MJ fixe un facteur de difficulté (action simple) ou effectue le jet pour le PNJ (opposition).
\item Le joueur lance l'intégralité des dés correspondants aux éléments choisis. Il choisit un résultat (en général le plus élevé) et l'annonce au MJ.
\item Le MJ annonce le résultat de l'action (réussite ou échec avec éventuellement des nuances).
\item Le joueur décrit l'action.
\end{itemize}

\chapter{En détail}
\section{Annonce de l'action}
Durant cette phase le joueur doit annoncer ce qu'il désire réaliser. L'important ici n'est pas de décrire avec précision le "comment" mais uniquement l'objectif de l'action. Le système de jeu va permet d'évaluer la réussite de l'objectif, le "comment" n'est qu'affaire de description et vient donc après le jet.

\section{Choix des éléments utilisés}
Le maître du jeu doit ensuite fixer les éléments de la fiche de personnage qui vont être utilisés. En général le maître du jeu procède ainsi :

\begin{itemize}
\item Il annonce la caractéristique utilisée en fonction de l'action entreprise. S'il hésite entre deux caractéristiques (cas assez rare), il prend alors la caractéristique la plus avantageuse pour le personnage.
\item Il annonce alors le domaine de compétence à utiliser (discrétion, natation, ...). Le joueur lui propose alors la compétence qu'il juge la plus appropriée. Le MJ est alors libre de l'accepter ou de la refuser. Si le joueur pense n'avoir aucune compétence, alors il fera le jet uniquement avec sa caractéristique ou ses équipements.
\item Si le joueur veut mettre en avant le coté négatif d'un trait, il l'indique alors au maître du jeu.
\end{itemize}

\section{Définir la difficulté}
Le MJ doit définir un seuil de difficulté vis à vis de l'action entreprise par le personnage. Ce seuil de difficulté est "indépendant" du personnage qui l'entreprend. Les contraintes dues à l'environnement (difficulté de concentration, temps limité, ...) ne sont pas à prendre en compte dans la difficulté. Elles seront exprimées autrement (voir chapitre "Les Faiblesses").

L'échelle de difficulté (indicative) que nous conseillons est la suivante :

\begin{itemize}
\item Action enfantine : 1
\item Action facile : 3
\item Action malaisée : 5
\item Action difficile : 8
\item Action héroïque : 11
\item Action quasi impossible : 13 ou plus.
\end{itemize}

\section{Réussite et échec des actions}
Tout n'est pas binaire dans la vie. Il y a des nuances dans la réussite ou l'échec d'une action entreprise par quelqu'un. Il en est de même dans la réussite des actions tentées par les personnages. 

\begin{itemize}
\item Réussite Totale : L'action qu'entreprend le personnage est réussie, sans aucun souci.
\item Réussite Partielle : L'action qu'entreprend le personnage est au choix réussie, mais avec des soucis, ou n'est pas réussie mais le personnage se rapproche de son objectif.
\item Réussite de justesse : L'action qu'entreprend le personnage est réussie, mais avec de graves soucis.
\item Échec Partiel : L'action qu'entreprend le personnage est au choix ratée, mais le personnage ne s'éloigne pas de l'objectif, ou alors elle réussit mais le personnage n'en tirera pas d'avantage pour se rapprocher de son objectif.
\item Échec Total : L'action qu'entreprend le personnage est ratée, de plus il s'éloigne de son objectif ou s'attire des ennuis.
\end{itemize}

On considère la réussite ou l'échec en fonction de la différence entre la difficulté et le résultat du jet sur cette échelle :

\begin{itemize}
\item Le résultat est de plus de 3 points supérieur à la difficulté : c'est une réussite totale.
\item Le résultat est entre 1 et 3 points supérieur à la difficulté : c'est une réussite partielle.
\item Le résultat est égal à la difficulté : c'est une réussite de justesse.
\item Le résultat est entre 1 et 3 points inférieur à la difficulté : c'est un échec partiel.
\item Le résultat est de plus de 3 points inférieur à la difficulté : c'est un échec total.
\end{itemize}

Ces seuils sont indicatifs. Avec l'expérience, vous apprendrez à faire peser chaque point, chaque marge différente, dans la balance pour évaluer le résultat d'une action.

\section{Description du résultat}
Une fois que le MJ a déterminé la réussite ou l'échec de l'action, c'est au joueur de décrire l'action en détails et son résultat. Le joueur est libre de la décrire et de l'interpréter comme il l'entend. Le MJ peut, toutefois, décider à tout moment de mettre son veto sur un élément particulier, ou de compléter la description.

La description doit bien sûr être en accord avec les résultats obtenus. Si ce n'est pas le cas, il est évident que le MJ doit la reprendre intégralement.

\exemple{Passer au dessus de barbelés}{
    Bob souhaite passer au dessus d'une grille en haut de laquelle se trouve des fils barbelés, pour entrer dans une maison bien gardée, en plus (et oui, c'est pas sport de découper la grille).

    Bref, cette action est incontestablement physique (où Bob a d8). Bob est un mec qui "Grimpe tout ce qui peut se grimper" où il a d10 (Bob aime donner des noms rigolos à ses compétences). La grille fait 2m de haut, avec de l'espace entre les croisillons, l'escalade est facile, c'est de passer les barbelés sans encombre, la difficulté.

    Le MJ l'estime donc à 5. Bob lance alors un d8 et un d10. Le d8 donne 7 et le d10 6. Bob choisit donc le plus élevé des dés, donc le résultat de son jet est de 7.

    Enfin, on compare le résultat de Bob à la difficulté fixée par le MJ. Il y a une différence de 2, c'est donc une réussite partielle.
}

\exemple{Le passage au dessus de barbelés}{
    La tentative de Bob est donc une réussite partielle, explique le MJ au joueur. C'est à dire donc que soit l'action est réussie, mais il y a un souci, ou alors elle n'est pas réussie mais quelque chose se passe. Le joueur choisit donc de lui-même (avec l'accord du MJ) ce qui viendra qualifier sa réussite de partielle (par opposition à totale) et non le MJ (de manière unilatérale).

    Bob explique donc que son pantalon s'accroche aux barbelés du côté extérieur, ce qui le fait basculer la tête en bas. Il ajoute que ses mollets frottent les barbelés, ce qui lui occasionne des tas de petites entailles qui piquent. Il termine en expliquant qu'il perd donc un peu de temps  à se sortir de cette situation.

    Il aurait pu choisir de passer sans souci ni blessure, mais en se réceptionnant bruyamment, ce qui n'aurait pas manqué d'alerter les vigiles proches.

    Ou encore, il aurait pu expliquer qu'il retombait de l'autre côté de la grille, les barbelés accroché au pantalon. Il aurait alors eu la possibilité de l'enlever sur une petite zone en prenant ses précautions pour ne pas se blesser. Il pourra retenter, mais les barbelés ne seront plus là au moment délicat de passer au dessus de la grille.
}

\chapter{Porter assistance}
Parfois, plusieurs personnages travaillent ensemble à la résolution d'un objectif commun. Dans ce cas, un des personnages va réellement exécuter l'action finale, on le nomme alors le "Meneur". Les autres personnages viennent alors juste apporter leur aide à ce meneur, avec leurs compétences propres.

Quand un tel cas de figure se présente, seul le meneur va effectuer le jet permettant de savoir si l'action est réussie ou au contraire échouée. L'aide des autres joueurs va permettre d'apporter des bonus augmentant les chances de succès. Ce bonus se présente sous la forme d'un (ou plusieurs) dé(s) supplémentaire(s) correspondant(s) au(x) dé(s) de compétence de la (ou des) personne(s) assistant le meneur.

Pour évaluer la réussite de l'action, le meneur va lancer :

\begin{itemize}
\item son dé de caractéristique
\item son dé de compétence
\item le dé de compétence des personnages qui l'assistent
\end{itemize}

Tous les personnages peuvent décider d'utiliser des traits ou du stress sur cette action. Le reste de la résolution se passe de façon standard.

\chapter{Les traits}
Les traits d'un personnage peuvent intervenir en jeu pour toutes actions en rapport. Par exemple, un trait "Mercenaire de sa majesté" pourra intervenir sur les actions militaires, mais aussi sur les actions en rapport avec le statut du personnage.

Il existe deux types d'utilisation :

\begin{itemize}
\item Comme avantage : il est utilisé pour aider le personnage à accomplir son action.
\item Comme désavantage :il est utilisé pour mettre en difficulté le personnage, soit en le pénalisant dans une action, soit en amenant le personnage à se placer dans une situation difficile.
\end{itemize}

\section{Comme avantage}
Un trait peut être utilisé lorsqu'il correspond à l'action entreprise. L'utilisation ou non d'un trait peut être discutée entre les joueurs et le MJ.

Les traits peuvent être ceux du personnage qui peuvent l'aider dans l'action qu'il tente, mais également des éléments Descriptifs de l'environnement (des débris pour se cacher, un tuyau par terre comme arme improvisée, un lustre pour échapper au bretteur ennemi, etc...) !

\exemple{Exemples d'utilisation de trait}{

\begin{itemize}
\item Le trait Forte-Tête peut être utilisé pour résister à une tentative de Persuasion. Par contre, il ne peut pas être utilisé pour gagner une course de vitesse (sauf si le joueur trouve vraiment une explication béton, dans ce cas, acceptez-la car il faut favoriser l'inventivité des joueurs).
\item Le trait Agilité Féline peut être utilisé pour tenter une acrobatie périlleuse, pas pour effectuer un travail mental !
\end{itemize}

}

Un trait utilisé permet de :

\begin{itemize}
\item Relancer tout ou partie d'un jet de dés
\item Obtenir un bonus de +2 sur le résultat total
\end{itemize}

Une fois utilisé, le joueur doit effectuer une coche à coté du trait. Cette marque permet d'indiquer que le trait a déjà été utilisé. Il ne pourra plus l'être par la suite sauf si le joueur obtient une décoche (cf Regain de trait) ou si il dépense un point de Stress.

Dans certains univers à l'ambiance plus héroïque, le maître du jeu pourra autoriser le joueur à cocher 2 fois chaque trait avant de le rendre inutilisable.

\section{Comme désavantage}
Un trait peut également être utilisé pour pénaliser le personnage : 

\begin{itemize}
\item soit en amenant le personnage à se mettre en difficulté : foncer sans réfléchir sur le grand méchant à cause de sa haine pour lui, plonger dans un piège à cause de son insouciance, ou tout simplement attirer les suspicions à cause de ses origines.
\item soit en pénalisant le personnage lors d'une action : sa peur du vide qui lui rend difficile la traversée d'un pont qui passe au dessus de la vallée, sa fascination pour les femmes qui l'empêche de repérer le mensonge dans la bouche d'une jolie dame, ... Dans ce cas, le joueur réduit tous les dés lancer d'un échelon (D6 devient D4, D10 devient D8, et D4 disparaît).
\end{itemize}

Si la mise en avant du trait comme désavantage à un intérêt scénaristique, le MJ vous gratifiera d'une décoche (possibilité de décocher un trait déjà coché). 

\section{Décocher un trait}
Il existe trois façon d'obtenir des décoches permettant de réutiliser à nouveau un trait qui avait été coché :

\begin{itemize}
\item Utiliser un désavantage (comme expliqué au chapitre précédent).
\item Effectuer une interprétation exceptionnelle de votre personnage. Quand vous jouez votre rôle avec soin et que vous participez à établir une bonne ambiance autour de la table, le maître du jeu est incité à vous récompenser par une décoche immédiatement. Ce gain devra toutefois rester exceptionnel et gratifier un réel effort de la part du joueur.
\item Lors des scènes de repos (longue nuit de repos, repas tranquille dans l'auberge du coin, ...), le maître du jeu offrira une décoche pour les joueurs ayant contribué à faire avancer le scénario dans les scènes précédentes. Si vous avez bien interprété votre personnage, que vous avez effectuez quelques actions utiles ou soumis de bonnes idées, vous serez donc récompensé.
\end{itemize}

\chapter{Le Stress}
Les points de Stress sont une réserve de points directement à disposition des joueurs. Tous les joueurs peuvent, quand ils le désirent, utiliser des points de Stress. Une fois les points de Stress utilisés, ils sont mis à disposition du MJ. Le MJ pourra alors les utiliser pour ses PNJ et les remettra ensuite dans la réserve générale.

En général, le nombre de points de stress disponible est égal à deux fois le nombre de joueurs présents. Le maître du jeu pourra bien sûr faire varier cette réserve selon l'ambiance qu'il désire obtenir.

Les points de Stress ont différentes utilisations :

\begin{itemize}
\item Utiliser un trait déjà coché
\item Faire abstraction des ses blessures
\item Influencer la narration
\item Utiliser une compétence complémentaire
\end{itemize}

\section{Utiliser un trait déjà coché}
Un joueur peut avoir envie d'utiliser un trait qu'il a malheureusement déjà coché. Les points de Stress lui permettent de se dépasser en utilisant le trait malgré la coche.

L'utilisation du trait se passe comme une utilisation standard.

Attention toutefois, un trait ne peut être utilisé qu'une seule et unique fois sur une même action : impossible de dépenser plusieurs point de stress sur le même trait, ou de dépenser un point pour utiliser un trait qui vient tout juste d'être coché.

\section{Influencer la narration}
Cette option doit être validée par le maître du jeu avant utilisation. Nous incitons toutefois les maîtres du jeu à l'autoriser car elle donne aux joueurs un peu plus de pouvoir et permet parfois des rebondissements originaux. Bien sûr, à chaque utilisation, le MJ est libre de donner son veto.

Influencer la narration permet aux joueurs de dépenser des points de Stress pour compléter la description du MJ. Il ne s'agit pas de venir contredire ce que le MJ a déjà dit, mais bien de préciser des choses qu'il n'a pas encore dites. L'influence doit être quelque chose de subtil et discrète : une modification trop flagrante est à refuser.

Un nouvel élément Descriptif de scène peut être créé suite à la description du joueur. Dans ce cas, la première utilisation que le joueur fera de ce nouveau Descriptif de scène est considérée comme gratuite.

\exemple{Exemple d'influence sur la narration}{

\begin{itemize}
\item Le MJ décrit que le joueur, passant de ruelle en ruelle, tombe soudain sur un cul de sac. Le joueur peut alors dépenser 1 point pour indiquer qu'une échelle est posée le long du mur, et que, grâce à elle, il peut se sortir de cette impasse !
\item Coincé dans une cave, le joueur explique au MJ qu'il compte tenter d'ouvrir la porte avec une pièce de métal trouvée dans un recoin sombre de la pièce.
\end{itemize}

}

\section{Faire abstraction des ses blessures}
Cette option permet au joueur de ne pas être pénalisé par ses blessures pour une action donnée. En dépensant un point de stress, le joueur peut effectuer son jet sans pénalité. L'annulation ne vaut que pour l'action en cours : la faiblesse pénalisera à nouveau le joueur dès la prochaine action.

\section{Compétence complémentaire}
Parfois, un personnage possède une compétence qui peut l'aider à accomplir une action, sans toutefois être la compétence principale. On appelle cette compétence une compétence complémentaire. 

Si le joueur veut faire intervenir cette compétence complémentaire dans un jet de dé, il peut le faire en dépensant un point de stress. Il ajoute alors le dé de la compétence en plus de ces dés classiques, comme il l'aurait fait dans le cas de "Porter une assistance".

\exemple{Porter assistance et compétence complémentaire}{
    John est un "commandant" (d6) qui est "habitué à travailler avec de l'infanterie" (d10). Il mène un assaut face à un ennemi.

    Un vaisseau mené par Bob survole la zone et la scanne. Bob utilise donc ses compétences en "senseurs" (d8) pour isoler les informations intéressantes et les envoyer à John.

    De son côté, John, fier de sa "grande expérience dans les tactiques d'infanterie" (d10), essaye de donner les ordres les plus pertinents.

    La compétence principale de cette action correspond à la compétence de commandement de John (d6). John décide de s'aider avec les données transmises par Bob. L'action de Bob est donc considérée comme une assistance. Il décide également de profiter de son expérience en infanterie et dépense un point de stress. Il va donc lancer sa caractéristique de Social (d4), sa compétence de Commandement (d6), sa compétence d'infanterie (d10) et la compétence de senseur de Bob (d8). Par contre, il aura déjà dépensé un point de Stress.

}


\part{Conflits}
\chapter{Types et déroulement des conflits}
Dans beaucoup de jeux de rôle, quand les Personnages Joueurs (PJ) rencontrent un Personnage Non Joueur (PNJ) agressif, généralement, cela se solde par un conflit. Mais ce conflit n'est pas forcément physique. En effet, tout dépend de ce que les joueurs souhaitent faire. S'ils interrogent le PNJ, cela pourra se régler en un conflit d'Intellect, pour représenter le duel de volonté entre l'interrogateur qui veut l'information, et l'interrogé qui ne veut pas la donner.

Qu'ils soient physiques, mentaux, sociaux, d'influence, nous gérons dans Fusina tous les conflits de la même manière, mais avec deux degrés de détails : les conflits courts et les conflits importants.

\section{Les conflits courts (contre les figurants)}
Ces conflits sont les plus courants : il s'agit de toutes les situations où un personnage affronte (physiquement, socialement, mentalement, etc...) quelque chose (un ennemi, une tâche complexe,...). On ne se préoccupe que de l'action du personnage. En termes de jeu, le joueur indique au MJ l'action qu'il souhaite accomplir.

En fonction de l'action souhaitée, le MJ détermine la difficulté, et fait faire un seul jet au joueur pour déterminer l'issue de la scène. En effet, les conflits courts sont là pour l'action, mais n'ont pas de réelle importance dans l'histoire, la scène est donc résolue en une fois.

\section{Les conflits importants}
A l'inverse, le(s) grand(s) méchant(s) compte(nt) beaucoup dans l'histoire, ils ne peuvent pas apparaître uniquement 30 secondes ! Il y a également certaines scènes porteuses d'une forte intensité dramatique qui méritent d'être traitées plus longuement.

Dans ce cas, les actions vont s’enchaîner en faisant progresser les différents objectifs poursuivis par les intervenants. Un but est rarement atteint en une seule action. Il faut en général plusieurs réussites pour réaliser un objectif. 

\chapter{La résolution d'un conflit long}
\section{Définition des termes de bases}
\section{Déroulement}
Le conflit est découpé en round. La durée d’un round dépend du type de conflit :

\begin{itemize}
\item Un round de combat va durer une dizaine de seconde
\item Un round d’une bataille va durer plusieurs minutes
\item Un round social pourra durer de longues minutes
\item Un round dans un combat d’influence va durer plusieurs heures, voir même plusieurs jours !
\end{itemize}

Lors d’un round, chaque protagoniste va avoir l’occasion d’effectuer une action. Les actions se résolvent selon la règle standard de Fusina. Quand une action va à l’encontre d’un personnage, celui-ci peut se défendre. L’action est alors résolue selon les règles d’opposition simple.

Lors des oppositions, certains personnages vont subir des handicaps, aussi appelés blessures dans le cas de combat. Ces handicaps vont pénaliser le personnage au cours de l'action, mais aussi pour le reste de l'aventure (cf chapitre Handicap et Blessure). 

\section{Qualificatif de scène}
L'environnement d'une scène donnée peut également offrir aux joueurs des éléments qu'ils pourront utiliser, ou au contraire subir. Ces éléments sont représentés par les "qualificatifs de scènes". C'est au maître du jeu qu'incombe la lourde tâche d'informer les joueurs sur les possibilités de la scène. Bien sûr les joueurs sont encouragés à faire des propositions au MJ en fonction de sa description ! Un qualificatif peut à le fois être utilisé comme faiblesse, et comme trait.

Prenons quelques exemples :

\begin{itemize}
\item Être en apesanteur pour des personnages non habitués est un handicap. Ce handicap disparaîtra bien sûr dès que le personnage y aura remédié d'une façon ou d'une autre (bottes magnétiques, scaphandre avec propulseur, ...).
\item L'obscurité peut être à la fois un handicap (pour viser par exemple) et un trait (pour être discret).
\item Un position dominante sur un champ de bataille est un trait en ce qui concerne la visée, ou la direction des troupes.
\end{itemize}

\chapter{Blessures, handicap, et mort}
Dans les films (oui encore), il n'y a pas de jauge de santé au-dessus de la tête du personnage. On peut voir ses blessures, on voit que chaque grosse blessure le handicape et on déduit donc s'il est encore en état ou pas d'agir, s'il est proche ou pas de la mort.

Pour refléter ceci, nous allons utiliser un niveau de santé. Ce niveau représente l'état global du personnage sans détailler si il s'agit d'une ou plusieurs blessures. Le type des handicaps reste donc purement narratif.

\section{Subir des handicaps}
Un personnage peut subir un handicap lorsqu'il perd une opposition durant un conflit. Subir un handicap signifie que le niveau de santé du personnage va diminuer d'un ou plusieurs niveaux.

Le nombre de niveaux perdus par le personnage dépend de la marge d'échec du personnage : un niveau perdu pour 3 points de marge. Ainsi, un personnage qui perd une opposition d'1 point perdra uniquement un niveau. Si il la perd de 4 points, il perdra un second niveau.

Les niveaux de santé sont les suivants :

\begin{itemize}
\item Indemne
\item Égratigné (pas de malus)
\item Blessé (-1 échelon aux dés)
\item Gravement blessé (-1 échelon aux dés)
\item Mortellement blessé (-2 échelons aux dés)
\item KO

Le joueur commence bien sûr indemne. A chaque niveau de santé correspond un malus indiqué dans la liste ci-dessus. Ce malus est exprimé en échelons sur tous les dés lancés.

Par exemple, un personnage qui doit lancer D8 et D6 mais qui est blessé effectuera un jet avec D6 et D4, si il est mortellement blessé il lancera uniquement un D4.

Quand le joueur arrive à KO, c'est au maître du jeu de décider les effets exacts : inconscience, mort si jamais personne ne vient le soigner vite, mort sur le coup, ...

\exemple{Exemples de blessures}{

    Bob souhaite sauter par dessus une grille avec des barbelés pour entrer dans la banque. S'il ne fait pas une réussite totale, il se prendra des blessures en s'éraflant sur les barbelés.

    Une manière de procéder pour le MJ serait de dire qu'en cas de réussite partielle, il parvient à passer mais s'érafle légèrement, c'est donc une blessure légère. En cas d'échec partiel, soit il ne passe pas et se fait peu d'éraflures (pas de blessure) soit il passe mais s'érafle profondément à plusieurs endroits aux pieds (perte d'un niveau de santé).

}
\end{itemize}

\chapter{Et si les ennemis sont nombreux ?}
Le système de conflit permet de gérer une simple escarmouche mais doit également être utilisable pour un groupe de plus grande importance ou même un combat de masse.

Dans cette situation, les petits groupes sont gérés par un dé bonus supplémentaire dépendant de l'avantage numérique du groupe sur le ou les personnage(s) et de la cohérence de ce groupe. Le dé se détermine en fonction de ces deux critères. Le dé de base est déterminé par l'avantage numérique et sera affecté par la cohésion de ce groupe, en bien ou en mal.

Voici le niveau du dé en fonction de l'avantage numérique du groupe sur le(s) personnage(s) :

\begin{itemize}
\item Léger avantage : d4.
\item Avantage moyen : d6.
\item Avantage important : d8.
\item Avantage écrasant : d10.
\end{itemize}

On prend en compte ensuite la cohésion du groupe :

\begin{itemize}
\item Groupe pas du tout coordonné, pas d'habitudes de fonctionnement en groupe : Le dé bonus baisse d'un échelon.
\item Groupe à cohésion classique, quelques belles ententes mais pas d'entraînement optimisant tout ça : On ne change pas le dé.
\item Groupe à forte cohésion, chaque membre du groupe a l'habitude de fonctionner dans le groupe : Le dé bonus augmente d'un échelon.
\end{itemize}

Cette gestion de groupe ne s'applique pas seulement à des conflits physiques mais à tous types de conflits, que cela puisse être un conflit social, mental ou d'influence...

Le niveau de santé existe toujours mais il représente l'état global du groupe et non celui d'un personnage seul.

\note{Et pour les groupes encore plus grands ?}{

    Et bien, continuez à gérer cela comme des groupes de personnages !

    Après tout, l'échelle change, mais soit il s'agira d'un conflit "classique", soit l'un des camps n'a aucune chance et l'autre le surpasse complètement. Et à ce moment là, on peut gérer cela comme un conflit contre des figurants, mais à grande échelle. 

    Bref, il n'y a pas besoin d'une gestion différente des conflits !

}

\chapter{Conflits sociaux}
Imaginons que le joueur soit un politicien qui a eu plusieurs aperçus de l'existence d'une vie extraterrestre sur Terre. Il doit convaincre plus haut que lui de la menace, mais n'a que peu de preuves directes en dehors du fait d'avoir été témoin. Il parvient, en faisant jouer ses contacts, à obtenir une audience auprès du chef de l'état de son pays.

Son but est donc de convaincre le chef d'état d'agir et de prendre les devants sur cette menace. 

Dans les jeux de rôle "classiques", beaucoup de MJ vont jouer la scène au roleplay, et faire un petit jet à la fin de la scène pour voir si finalement le RP sert à quelque chose.

Et bien ici, non. Comme toute action importante dans Fusina, le résultat des dés (réussite ou échec) doit être décrit par le joueur. Ici, le MJ va indiquer si le joueur réussit ou échoue, partiellement ou complètement comme dans toute action normale.

Aux joueurs et au MJ d'arriver au consensus. Pour illuster, quelques exemples appliqués à la situation décrite ci-dessus :

\begin{itemize}
\item Échec total : Le chef d'état ne croira pas du tout le politicien, et va sûrement agir pour que ce politicien perde en crédibilité.
\item Échec partiel : Le chef d'état ne croira pas du tout le politicien, et va sûrement lui demander de revenir quand il aura des preuves.
\item Réussite partielle : Le chef d'état aura des doutes, et donner des ressources au politicien pendant une durée limitée, pour obtenir des preuves avant d'agir plus ouvertement.
\item Réussite totale : Le chef d'état croit complètement le politicien, et lancera toutes les procédures qu'il faut pour annihiler la menace.
\end{itemize}

Ensuite, le joueur explique l'attitude qu'aura le chef d'état au fur et à mesure du dialogue au MJ.

Enfin, les intervenants jouent la scène, en une seule fois, sans jets de dés au milieu, dont l'issue est celle décidée plus tôt avec le MJ.

\chapter{Créatures et Adversité}
Le rôle de maître du jeu est de décrire et de faire vivre l'univers, mais également les personnages qui le composent. Mais comment procéder quand le maître du jeu a besoin d'évaluer la compétence d'un personnage non joueur ? Comment procéder quand les joueurs doivent faire un jet en opposition ? Faut-il définir intégralement le personnage ? Pour la plupart des cas, la réponse est non ! 

Nous allons proposer ci-dessous plusieurs méthodes dépendant principalement du type de personnage.

\section{Les Figurants}
Les figurants sont des personnages qu'on croise dans la partie, mais qui n'ont aucun rôle réel à jouer. Les figurants ne représentent même pas un défi pour le personnage, ils sont juste là pour faire joli, être des pots de fleurs ! Par exemple, l'herboriste qui vend ses herbes et onguents sur le marché est un figurant !

Que se passe-il si un joueur rentre en opposition avec un figurant ? La réponse est simple, il n'y a pas d'opposition ! Le figurant est un élement du décor et doit être traité comme tel. Le joueur va donc réaliser son jet contre une difficulté fixée par le maître du jeu, exactement comme s'il tentait juste d'escalader un mur.

\section{Les sbires}
Les sbires sont un peu plus que des figurants. Ils n'ont guère d'importance dans le déroulement du scénario mais sont tout de même là pour représenter un défi à relever, mineur certes, mais un challenge tout de même. Les sbires ont en général besoin d'être en nombre pour représenter un danger.

Comment connaître leurs caractéristiques ou leurs compétences ? Pour un sbire, le maître du jeu va uniquement définir des caractéristiques, et une occupation. 

Les caractéristiques seront définies en répartissant 4 echelons dans les caractéristiques. Exceptionnellement le maître du jeu pourra glisser 1 échelon de dé d'une caractéristique vers une autre pour spécialiser davantage le sbire. 

L'occupation correspond à l'activité principale du personnage. Il s'agit en général de son métier. Lors d'une action, l'occupation permettra de déterminer le dé utilisé pour la compétence. 

\begin{itemize}
\item Si le jet correspond directement à l'occupation du sbire, alors la compétence vaut D8.
\item Si le jet est seulement en rapport avec son occupation, alors la compétence vaut D6.
\item Dans tous les autres cas, on considère que le sbire n'a pas de compétence associée.

Concernant la dépense de Stress pour le maître du jeu, il est possible de dépenser uniquement 1 point pour un jet concernant l'occupation du sbire. Dans tous les autres cas, pas de stress pour les sbires ! 

Prenons un exemple : des brigands. Mes brigands vont avoir les valeurs suivantes : Physique : D8, Ame : D6, Social : D6, Intellect : D4, Influence : D4. Quand il s'agit de tendre une embuscade et de faire un peu de combat, mes brigands auront D8 en compétence. Pour fuir à travers la forêt, ils auraient un petit D6. Pour déjouer un piège mécanique, là, ils devront se contenter de leurs caractéristiques.

+++Les Lieutenants+++

Un lieutenant est le chef d'un groupe de sbires. C'est lui qui commande et organise sa petite troupe et la mène à l'action. Le lieutenant se génère et s'utilise de la même manière qu'un sbire à deux particularités près :

\item Un lieutenant peut utiliser 2 points de Stress. Il peut les dépenser dans les jets directements liés à son occupation, mais aussi dans les jets en rapport.
\item Quand un groupe de sbires perd son lieutenant, il est à la déroute et travaille moins efficacement. Le dé de groupe perd donc un échelon.
\end{itemize}

\section{Les PNJ principaux}
Les PNJ principaux sont les personnages qui ont un vrai rôle à jouer dans l'aventure et qui sont l'égal, ou presque, des personnages. Ce sont les grands méchants, mais aussi les seconds rôles permettant de générer une ou plusieurs scènes intéressantes.

Pour les PNJ principaux il faut définir un niveau de tension. En général le niveau de tension vaut 2, voir 3 pour les PNJ vraiment dangereux ou importants. 

Pour la génération, le PNJ disposera donc de (4 + Tension) échelons à répartir pour ses caractéristiques. On lui attribuera ensuite 2 occupations. L'occupation principale sera à d10, tandis que la deuxième sera à d8. Cela fonctionne ensuite comme pour les sbires et les lieutenants, les jets directement en rapport se font avec le dé de la compétence, les jets en rapport avec le dé de l'échelon en-dessous, et on ne prend pas le dé de la compétence si l'action n'a rien à voir avec elle.

\section{Les Bêtes}
Par bêtes on entend toutes les créatures non-intelligentes et non-pensantes, que ce soit des animaux, des monstres, ou des créatures de légendes. Pourquoi faire une catégorie particulière pour les créatures ? Tout d'abord, les caractéristiques définies pour les personnages ne sont pas adéquates pour un animal. L'influence a-t-elle le moindre sens pour une bête ? De plus, les bêtes ne peuvent pas utiliser d'armes, il faut donc leur donner d'autres moyens de rester un défi.

Encore une fois nous parlons des bêtes censées présenter un réel défi : les autres bêtes seront gérées comme des figurants. 

Pour les bêtes nous allons donc définir trois caractéristiques :

\begin{itemize}
\item Physique : concerne toutes les actions nécessitant de la force, de l'endurance, de l'agilité, ou de la résistance.
\item Instinct : capacité à faire confiance à ses instincts, à ses perceptions.
\item Tenacité : capacité à lutter contre ses instincts, ses peurs, ses phobies. 
\end{itemize}

Leur génération est ensuite identique aux sbires, lieutenants, ou PNJs principaux.

\note{Je veux un PNJ plus puissant !}{

	\begin{quotation}
Vous désirez créer un PNJ plus costaud ? 
	\end{quotation}

	\begin{quotation}
Nous vous proposons cette solution qui donnera à votre PNJ un plus large panel de possibilités, tout en le rendant plus intéressant. Vous pouvez ajouter à ce PNJ un champ de compétence supplémentaire, qui sera considéré comme une seconde occupation. En échange, vous devez lui ajouter une faiblesse qui pourra être exploitée par les joueurs. La valeur de la nouvelle occupation sera égale à la valeur des occupations principales, diminuée d'un échelon.
	\end{quotation}

	\begin{quotation}
Vous pouvez également augmenter d'un échelon une occupation existante. Le coût est identique, vous devrez rajouter une nouvelle faiblesse à votre création.
	\end{quotation}

	\begin{quotation}
Par exemple, je décide de faire un loup boosté aux hormones. De base le loup à un champ de compétence "Chasse" à d8. Je décide de lui rajouter "Affoler les animaux tels que les chevaux" en champ de compétence, prévoyant de faire chuter les cavaliers. Je peux donc la mettre à d6 (l'échelon en dessous de d8), en échange j'impose à mon loup une faiblesse "Phobie du feu".
	\end{quotation}

}

\note{Génération partielle}{

Les règles ci-dessus permettent de générer des personnages non-joueurs en quelques minutes (sauf PNJ principaux). Mais avez-vous vraiment besoin d'une définition aussi précise ? Souvent une seule caractéristique vous sera utile pour les sbires !

Au final, nous vous conseillons de générer vos sbires et lieutenants à la volée ! Décidez des valeurs de chaque caractéristique ou équipement au moment où vous en avez besoin, pas avant. Dans mon exemple de brigand, au début j'aurai défini uniquement le Physique. Quand un joueur tente d'intimider, là je me pose la question du score en Âme (d6 ou d4 ?). Avant cela, pourquoi s'embêter ?

}


\part{La Progression}
\chapter{Attribuer les points d'évolution}
Le système de progression de Fusina a pour objectif de refléter réellement ce que fait le personnage, comment il est perçu par le joueur, mais aussi par le maître du jeu et les autres joueurs. Ce système permet également de faire un debrief sur la partie.

Les points d'évolution se répartissent donc en plusieurs phases.

\section{Phase 1 - Le choix du joueur}
Le joueur choisit ce qu'il désire faire évoluer sur son personnage. Il attribue un seul et unique point d'évolution. Ce point peut concerner un trait, des compétences, une faiblesse, ou un équipement.

\section{Phase 2 - Le choix du groupe}
Le groupe de joueur choisit ensuite un endroit où attribuer un point pour le personnage. Pour cette répartition on se base sur les actions du personnages durant la partie : Qu'est-ce qui a été important chez le personnage durant cette aventure ? Qu'est-ce que le groupe a retenu ?

\section{Phase 3 - Le choix du maître}
Le maître du jeu attribue à son tour un seul et unique point au personnage. Il peut se baser sur le vécu du personnage durant la partie, mais aussi sur sa propre volonté et sur l'intrigue globale de sa campagne.

\section{Phase 4 - L'oubli}
Avec accord du maître du jeu, un joueur peut décider que son personnage "oublie" un élement qui le définit (trait, compétence, équipement, ...). En échange le maître du jeu lui redonnera un élement de même valeur. Parfois il arrive que c'est le scénario qui impose une perte (comme un équipement volé ou détruit). La conséquence est la même, le maître du jeu redonnera au personnage quelque chose de valeur identique.

\section{Phase 5 - L'accomplissement}
Cette phase n'est pas systématique. Elle intervient uniquement quand un personnage a accompli quelque chose qui lui était propre (abattre un de ses ennemis, atteindre un de ses objectifs). Dans ce cas le maitre du jeu attribue au joueur 1 à 5 points d'évolution. Le nombre de points dépend de l'importance de l'accomplissement.

\chapter{Faire évoluer le personnage}
\section{Compétence}
\begin{itemize}
\item Acheter un d4 : 1 point
\item Passer de d4 à d6 : 2 points
\item Passer de d6 à d8 : 3 points
\item Passer de d8 à d10 : 4 points
\item Passer de d10 à d12 : 5 points
\end{itemize}

\section{Trait}
Acheter un nouveau trait coûte 5 points.

\section{Lier un trait}
Lier un trait à une caractéristique standard ou occulte coûte 5 points.

\section{Faiblesse}
Supprimer une faiblesse du personnage coûte 10 points.

\section{Equipement}
Il est possible d'acquérir des équipements particuliers avec des points d'évolution. Les coûts donnés ci-dessous sont indicatifs, le MJ est seul juge du coût qu'il estime juste pour un équipement particulier.

\begin{itemize}
\item Equipement à d6 : 1 point
\item Equipement à d8 : 3 points
\item Equipement à d10 : 5 points
\item Equipement à d12 : 10 points
\end{itemize}

Ces équipements doivent avoir une identité propre qui les rend uniques.


\part{Pouvoirs occultes}
Il existe de nombreux types de pouvoirs occultes : super-pouvoirs, pouvoirs psychiques, magie, ...

Tous ont leur identité et leur importance dans l'univers qu'ils servent. Nous ne pouvons pas utiliser le même système pour un univers où les pouvoirs sont légions et communs, instinctifs et pour un univers où il s'agit d'une magie composée de pouvoirs uniques possédés par de rares individus.

Nous proposons donc 2 systèmes, modulables et légers :

\begin{itemize}
\item Gestion des pouvoirs par des compétences occultes
\item Gestion des pouvoirs par des caractéristiques occultes
\end{itemize}

Nous allons tout d'abord voir comment le Meneur déterminera la difficulté d'une action impliquant des pouvoirs occultes, puis chacune des deux méthodes pour gérer ces pouvoirs au niveau des règles.

\chapter{Déterminer la difficulté}
L'échelle de difficulté d'une action impliquant des pouvoirs occultes est la même que celle des autres actions. 

Le dosage est juste à faire au feeling selon l'univers et ce que le joueur souhaite faire. Nous ne fournirons pas de liste fixe, ni de tableaux à valeurs et paramètres, car pour nous, plus l'usage des pouvoirs occultes est complexe, moins les joueurs seront inventifs, se contentant d'utiliser toujours les quelques sorts qu'ils auront préparés...

Basez-vous sur ceci : si soulever des montagnes est facile dans un univers, la difficulté pour soulever une grosse pierre sera faible. Dans le cas contraire, elle pourrait être héroïque rien que pour soulever un caillou dans un univers où la magie est bien plus faible !

L'échelle de difficulté reste identique aux actions "classiques".

\chapter{Gestion des pouvoirs par des compétences occultes}
Ce système convient aux systèmes où les pouvoirs sont un art qui s'apprend, qui s'entretient, mais où tous ont la capacité de les maîtriser, et où c'est uniquement l'entraînement qui différencie le niveau de maîtrise.

Dans ce système, les pouvoirs sont représentés par des compétences dites "occultes". Le joueur choisit et attribue des points dans ces compétences de la même façon que pour les compétences non occultes. 

\section{Utilisation de pouvoir de manière "brute"}
Cette manière d'user de pouvoirs se déroule comme suit : 

\begin{itemize}
\item Le joueur fait un jet de la compétence adéquate avec la caractéristique Âme
\item La dépense de points de Panache ou Stress dépendent de la puissance du pouvoir. 1 point correspond à un pouvoir faible, 2 points à un pouvoir important, 3 points à un pouvoir faisant partie des plus puissants de l'univers.
\end{itemize}

\section{Utilisation de pouvoir en assistance}
Vous pouvez utiliser vos pouvoirs, selon les univers, de manière différente, par exemple pour vous aider lors d'une action (utiliser par exemple votre magie de l'air pour courir plus vite lors d'une fuite ou d'une chute).

Dans ce cas là, vous faites le jet classique de l'action (par exemple Physique + Athlétisme) mais vous lancez un dé de plus correspondant à la compétence occulte que vous utilisez pour vous aider. Vous dépensez donc un point de Panache/Stress pour utiliser la compétence de magie en assistance, comme vous le feriez pour une compétence normale.

\section{Caractéristique différente}
Si la caractéristique Âme ne correspond pas à votre idée de comment fonctionnent les pouvoirs ou la magie dans votre univers, vous pouvez tout à fait ajouter une caractéristique exprès pour ça, ou en prendre une autre (l'intelligence, par exemple). Dans ce cas, faites faire les jets avec cette caractéristique et la compétence qui va bien plutôt qu'avec Âme. C'est au MJ de juger encore une fois.

\chapter{Gestion des pouvoirs par des caractéristiques occultes}
Ce système convient aux univers où les pouvoirs sont un don lié au plus profond de son être. Les pouvoirs ne s'apprennent pas, ils sont innées.

Dans ce système, les pouvoirs sont représentés par des caractéristiques dites "occultes". Le nombre et le détail de ces caractéristiques dépend de l'univers choisi. 

Lors de la création, le joueur lie les traits qu'il a choisi à une des caractéristiques occultes. Chaque trait est donc lié à deux caractéristiques : une "classique" et une occulte.

\section{Utilisation de pouvoir de manière "brute"}
Cette manière d'user de pouvoirs se déroule comme suit : 

\begin{itemize}
\item Un jet de la caractéristique occulte concernée avec la caractéristique Âme, associé à une difficulté choisie par le maitre du jeu.
\item La dépense de points de Panache ou Stress dépendant de la puissance du pouvoir. 1 point correspond à un pouvoir faible, 2 points à un pouvoir important, 3 points à un pouvoir faisant partie des plus puissants de l'univers.
\end{itemize}

\section{Utilisation de pouvoir en assistance}
Comme pour la gestion par compétence, ici aussi, les pouvoirs occultes peuvent servir en tant qu'assistance sur des actions "classiques". Par exemple, si vous combattez un ennemi et que vous maîtrisez la magie de la terre, vous pourriez sur une passe d'armes tenter de rendre le sol meuble sous les pieds de votre adversaire. Auquel cas vous ajouteriez la caractéristique occulte comme assistance au jet de combat en dépensant un point de Panache/Stress.

\note{Apprentissage progressif}{
Afin de conserver un apprentissage progressif des pouvoirs avec cette solution, ou de diminuer la puissance de ces pouvoirs, il est possible de ne lier qu'une partie des traits aux caractéristiques occultes.

Un seul trait lié permet de créer des personnages ayant à peine conscience de leurs pouvoirs alors qu'une totalité des traits liée crée des personnages chargés de pouvoirs, potentiellement très puissants.

Lier un trait à une caractéristique de pouvoir demande 10 points d'expérience.
}


\part{Règles avancées - Véhicules}
Les Véhicules vont être classés en deux catégories : 

\begin{itemize}
\item Les véhicules standard
\item Les véhicules importants
\end{itemize}

Chacune des catégories répond à des règles spécifiques.

\chapter{Véhicules standard}
Ils n'ont pas d'identité propre et ne sont pas destinés à avoir un fort impact dans le jeu. Ce sont tous les moyens de transport qui servent pour une unique scène. Dans des univers non-technologiques ils dépendent tous de cette catégorie.

Dans Fusina, ils sont gérés comme des équipements et apportent donc un dé supplémentaire à lancer.

\chapter{Véhicules importants}
Ils sont uniques et réellement liés au groupe ou à l'univers. Ils font des apparitions fréquentes dans les scénarios du groupe. Ce peut-être par exemple le vaisseau spatial du groupe ou un mécha customisé par un personnage.

Ils sont représentés par trois caractéristiques : 

\begin{itemize}
\item Motorisation : utilisation du véhicule "à fond les manettes".
\item Systèmes : détection, communication, armements à distance, ...
\item Maniabilité : capacité de mouvement et de changement de direction, mais aussi la capacité à combattre au corps à corps.
\end{itemize}

Ils vont également être dotés d'un certain nombre de traits et de faiblesses.

Lors d'une action engageant un véhicule, nous allons donc utiliser : la Caractéristique et la Compétence de l'utilisateur + la Caractéristique du véhicule. En plus de ses propres traits, le joueur peut activer les traits de l'engin.

Un véhicule est également doté de deux scores d'armure :

\begin{itemize}
\item Coque : protège les dispositifs dans le but d'éviter les dommages importants.
\item Protection : protège les passagers. Ce score est inutile quand ceux-ci sont entièrement couverts par le véhicule comme par exemple dans un vaisseau spatial.
\end{itemize}

Ces deux scores utilisent la règle de protection standard et représentent le nombre de niveaux de blessure que l'engin est capable d'encaisser chaque scène.

\section{Créer un véhicule}
Les véhicules possèdent un nombre de traits variable qui dépend de la puissance désirée :

\begin{itemize}
\item Véhicule faiblard : 2 traits de base.
\item Véhicule standard : 4 traits de base.
\item Véhicule puissant ou high-tech : 6 traits de base.
\end{itemize}

Comme pour les personnages, il est possible de choisir une faiblesse pour prendre un trait supplémentaire. 

Les traits sont ensuite reliés aux différentes caractéristiques ce qui permet de déterminer leur score.

Il n'y a pas de règle concernant l'attribution de la coque et de la protection. Ces deux scores doivent être attribués en fonction de la taille et du type de véhicule.

\section{Améliorer/Personnaliser le véhicule}
Il est possible de modifier l'engin pour le rendre plus performant, plus résistant, ... 

Pour le faire, les joueurs devront sacrifier des points d'expérience. Ils peuvent sacrifier uniquement les points de la phase 1 (celle où eux mettent un point d'expérience où ils le souhaitent). Les coûts en expérience sont les suivants :

\begin{itemize}
\item Nouveau trait : 5 points
\item Lier un trait à une caractéristique : 5 points
\item Coque ou protection supplémentaire : 3 points
\end{itemize}


\part{Conseils de maîtrise}
\chapter{Les descriptions}
\section{Résoudre avant de décrire}
Les vieux routards ont un réflexe basique : ils décrivent leurs actions, ils les résolvent, puis le MJ décrit à son tour. Chaque action passe donc par deux phases de description, une de l'attendu du joueur (ce qu'il souhaite faire) et ensuite celle de l'action réellement faite par le personnage (le résultat). 

La plupart des jeux fixent la difficulté également en fonction des détails que donne le joueur de son action. Certains joueurs se retrouvent donc à éviter de trop décrire et de trop en rajouter de peur d'obtenir des malus qui pourraient les amener à échouer alors qu'ils essayent juste de réaliser une action spectaculaire, digne des films d'action.

Fusina est conçu pour casser ce schéma de la manière suivante :

\begin{itemize}
\item Le joueur annonce son objectif, afin que le MJ puisse fixer une difficulté et indiquer le jet au joueur
\item On résoud l'action, le MJ annonce de manière succinte le résultat de l'action au joueur.
\item Le joueur décrit l'action.
\end{itemize}

La seule description de l'action se fait une fois le résultat connu. La description peut donc se concentrer sur ce qui est vraiment le résultat réel de l'action. On gagne en fluidité, et en précision. De plus, les joueurs peuvent savoir à quel point en rajouter en fonction de ce que le MJ leur dit (car le MJ indique dans sa description concise si l'action est réussie, ratée, de justesse ou pas).

Toutefois, ce point me mène directement au second conseil.

\section{Laisser les joueurs décrire}
Point très important, le MJ doit casser son monopole de la description du résultat des actions. Le MJ devrait se contenter d'annoncer froidement et de manière concise le résultat "technique" d'une action pour laisser aux joueurs libre cours à leur imagination dans la description de ce qu'il se passe. Le MJ conserve alors un rôle d'arbitre. Il peut décider de contredire une description s'il juge qu'elle n'est pas en phase avec l'univers ou le résultat technique, la compléter pour donner de la consistance. Mais le narrateur de l'action reste bel et bien le joueur.

Quels sont les avantages de cette méthode ?

\begin{itemize}
\item La description colle davantage au personnage. Un réel style peut être attaché à chaque action du même personnage.
\item Les joueurs ont des idées que le MJ n'aurait pas eues.
\item Les joueurs peuvent décrire des actions dignes des films (le héros qui se pend au lustre pour tomber sur son adversaire par exemple) sans limitation. Le résultat est déjà connu, la description n'est que panache.
\end{itemize}

\section{Penser Cinématique}
Voilà, finalement, le conseil le plus important que l'on peut vous donner, et qui correspond vraiment à l'optique de Fusina : Penser cinématique.

Concrètement, qu'entendons-nous par cela ? Nous pensons que la partie de jeu de rôle doit être vécue par les joueurs comme le serait un film de l'ambiance choisie. Attention, cela ne veut pas dire que les joueurs sont spectateurs, mais tout simplement que les joueurs interprétent les personnages principaux d'un film, ils sont le centre de l'histoire. 

Dans vos descriptions (mais aussi dans celles que les joueurs vont être amenés à faire), essayez de toujours penser à comment cela serait représenté dans un film. Imprégnez-vous avant les parties de films appropriés à votre univers, imprégnez-vous des angles de caméra, des scènes spécifiques, des odeurs qui ressortent de l'écran. Réunissez tous ces élements et exploitez-les dans vos séquences.

Et comment faire pour que les joueurs utilisent aussi cette vision dans leurs descriptions ? Guidez-les ! Proposez-leur des élements au fur et à mesure de leurs descriptions, complétez-les dans ce sens. Petit à petit, ce type de description se fera naturellement pour eux. 

Ah, et un dernier conseil... Ne cassez surtout pas le rythme ! Un doute sur le système ? Faites simplement comme vous le sentez, Fusina est adapté à une gestion au feeling, alors lâchez-vous ! Vous ne savez pas comment interpréter la gêne occassionée par ce champ de fumée, demandez juste au joueur de noter sur leur feuille une faiblesse "fumée" qui disparaîtra dès qu'ils en seront sortis. Vous ne savez pas comment gérer l'aspect fatigue ? Pareil, faites noter une faiblesse "Fatigue", et hop, c'est reglé ! Faites comme vous le sentez, sans jamais interrompre l'histoire et son déroulement, sans jamais laisser votre ambiance ou le rythme retomber.

\chapter{La Création de Personnage}
\section{Posez des questions}
La création des personnages est un moment crucial pour leur "vie" à venir en jeu. Le background des personnages n'est pas uniquement fait pour servir de décor, d'historique. Il s'agit d'un ensemble de faits et d'acteurs qu'il faudra intégrer, coûte que coûte, au scénario. Alors, ne laissez pas les joueurs le construire seuls dans leur coin.

À la fin de la création des personnages, demandez successivement à chaque joueur de décrire son personnage. Et, dès que vous avez la moindre opportunité, harcelez-le de question ! Votre but ? Faire ressortir de son background des personnages, des amis, des rivaux, des ennemis, bref, toute personne qui pourrait apparaitre dans les scénarios d'une manière ou d'une autre. Certains personnages tenteront de nuir au pj, d'autre tenteront de l'aider, d'autre agiront comme des faiblesses capables d'attirer les pires ennuis au personnage, et enfin, d'autres gêneront le pj par leur simple présence, sans même réellement interagir dans le scénario. 

À chaque apparition, le personnage du joueur prendra de la profondeur.

\section{Faites-le en groupe !}
Ne créez jamais les personnages de façon isolée, faites-le en groupe. Laissez les autres joueurs intervenir dans la séance de questions. Ils poseront peut-être des questions auxquelles vous n'avez pas pensé. Parfois ils pourront lier leurs personnages par un élement commun. Ils auront des bonnes idées qu'ils pourront soumettre. Tout le monde y gagnera, et vous aurez un vrai groupe de personnages joueurs.

Et comment faire pour les choses que les joueurs veulent garder secret ? Aborder ces points, et uniquement ceux-ci, en solo avec le joueur. Mais évitez d'en avoir trop tout de même. Les personnages sombres et mystérieux ayant tous nombre de sombres secrets, c'est un peu surfait !


\part{Adapter un univers}
Adapter un univers pour Fusina est chose relativement aisée. Pourquoi ? Tout simplement car le gros des choses spécifiques va se retrouver dans les compétences et dans les traits, points qui sont de toute manière libres. Que reste-t-il donc à faire ? Nous allons vous l'expliquer dans cette partie.

\chapter{Ajustement des traits}
Choisir le nombre de trait que possède un personnage à la création et le nombre de coche possible est extrêmement structurant pour l'ambiance de jeu. Vous n'aurez réellement pas le même feeling en jeu si vous pouvez effectuer deux coches ou une seul sur chaque trait !

Une coche par trait correspond à une ambiance aventureuse où les personnages sortent du lot sans être toutefois exceptionnels. En autorisant deux coches, les personnages deviennent de véritables héros capables de se dépasser pour accomplir des exploits. 

Le nombre de traits est plus délicat à faire varier. En diminuant ce nombre, les personnages deviennent commun, leurs spécificités ressortent moins. Ne descendez jamais en dessous de 4 traits, sinon vous risquez de nuir au plaisir des joueurs. Au délà de 6 traits, il devient difficile d'être original. N'employer cette solution que si certains traits représentent des pouvoirs ou des capacités directement issus de l'univers.

\chapter{Gérez vos peuples}
Comment voulez vous gérer vos peuples/races ? Un simple trait au nom du peuple ? Au contraire, des races imposant certains traits ou faiblesses ? A vous de voir ce que vous désirez, les deux solutions sont bonnes.

\chapter{Gérez votre magie}
Voici le point que jugé le plus difficile : gérer la magie. Tant de solutions, tant de particularités, bref, chaque jeu demande quasiment un système unique. 

Mais posez-vous déjà la question du type de magie ! La magie est-elle une magie de savoir, demandant un long apprentissage ? Si oui, optez pour la magie par compétence (ou une variante basée sur cette solution).

Vous avez au contraire une magie instinctive, naturelle ? Dans ce cas, basez-vous sur la solution de magie par caractéristique.


% LaTeX2e code generated by txt2tags 2.5 (http://txt2tags.sf.net)
% cmdline: txt2tags -t tex Fusina.t2t
\end{document}
