\clearpage

\chapter*{Crédits}

\vskip 140pt

\begin{center}

\textbf{Conception}

Lionel "Armaklan" Zuber

\vskip 80pt

\textbf{Avec la participation de}

Lionel "Armaklan" Zuber

Simon "Angeldust" Li

Sébastien "Segle" Gélé

Léandre "Salanael" Bernier

\textbf{Et remerciement à \ldots}

\end{center}

\clearpage

\chapter*{Licence}

Fusina est un \textbf{système générique de jeu de rôle}, sous licence \textbf{Creative Commons BY-SA} dont le détail est à cette adresse : \url{http://creativecommons.org/licenses/by-sa/3.0/fr/deed.en}.

Cette licence vous autorise à diffuser, modifier ce document, du moment que vous citez son auteur et que vos modifications soient sous cette même licence. C'est une licence virale.

Les icônes des encadrés font partie du pack d'icônes "ecqlipse 2" de chrfb, dont la galerie DeviantArt se trouve ici : \url{http://chrfb.deviantart.com/gallery/}. Elles sont en Creative Commons By : \url{http://creativecommons.org/licenses/by/3.0/deed.fr}.

Les bordures et images entourant les titres des chapitres sont des réalisations d'Ivy-Poison (Random Borders et Dirty Lines) sur Deviantart, dont la galerie se trouve là : \url{http://ivy-poison.deviantart.com/gallery/}.

Toutes ces images sont la propriété de leurs auteurs respectifs et sont utilisés ici dans le cadre des autorisations d'utilisation indiquées par les auteurs. Avant toute réutilisation, assurez-vous de bien respecter les conditions d'utilisation de leur travail.

\bigskip

La dernière version de Fusina est disponible à l'adresse suivante :

%\medskip

\begin{center}
  \Large\url{http://fusina-jdr.org}
\end{center}

%\bigskip

\vfill

\begin{center}
  \Large Bonne lecture !
\end{center}


\vskip 20pt

\clearpage

\chapter*{Guide de Lecture}

Pour faciliter la lecture, certaines zones ont été ressorties sur fond coloré. Ces zones illustrent ce qui est écrit dans le reste du document. Un petit logo en haut à droite de ces zones, ainsi que leur couleur de fond, indiquent leur type.

\exemple{Exemple}{
    Ce type de zone indique une ou plusieurs mises en situation illustrant un point (de règle ou non) défini au dessus.
}

\note{Note de conception}{
    Ce type de zone indique une note de conception. Les notes de conceptions servent à expliquer pourquoi nous avons fait tel ou tel choix, tant au niveau des règles que de l'univers. Elles permettent au lecteur de mieux sentir l'objectif présent derrière un élement.
}

\option{Règle optionnelle}{
    Ce type de zone indique des règles facultatives, pouvant mieux convenir que les règles "de base" dans certains types de parties par rapport à l'univers choisi par le MJ pour ces parties.
}

\remarque{Conseil}{
    Ce type de zone contient des remarques, des astuces ou des avertissements pour que le futur MJ puisse éviter certains problèmes que nous avons nous mêmes déjà rencontrés. Il s'agit principalement de Conseil donné au MJ de manière à l'orienté et à l'aider à utiliser Fusina dans l'optique que nous lui avons choisit.
}
